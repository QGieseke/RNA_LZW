\documentclass[letterpaper, 12pt]{artikel3}

 \usepackage{fullpage}
\usepackage{amssymb}
\usepackage{algorithm}
\usepackage{amsmath}
\usepackage{mathtools}
\usepackage{verbatim}
\usepackage{algpseudocode}
\usepackage{commath}
\usepackage{hyperref}
%\usepackage{yplan}


\usepackage{tikz}
%\usetikzlibrary{automata, positioning}
%\usepackage{algorithmic}

\title{CSC482B Project Proposal: RNA\_LZW (title subject to change)}
\date{\today}
\author{Quinn Gieseke, Claire Champernowne}

\begin{document}
\maketitle

% leaving this part in as a little format guide for how latex works if you need a refresher

% if you need latex tools, check out TeX Live https://www.tug.org/texlive/
% or MiKTeX https://miktex.org/
% there should be no difference for the simple stuff we will be doing


%\section{Prove the language $L = \{<M> | M $ when started on the blank tape, eventually writes a \$ somewhere on the tape $\}$ is undecidable. Use the undecidability of $A_{TM}$ to do this.}

%\textbf{Theorem:} $L$ is not decidable.

%\textbf{Proof:} For any $M,w$ we let $M_1$ be the TM which takes input string $x$:

%\begin{enumerate}
%	\item If $x \neq w$, $M_1$ rejects and writes no output.
%	\item If $x = w$, run as $M$ on input $w$, except for every instance where $M$ uses $\$$, replace with $\$'$, and if $M$ were to accept, first write \$ to the tape, then accept. 
%\end{enumerate}

%Now we construct TM $S$ to decide $A_{TM}$. Let $R$ be the hypothetical TM which can decide $L$:


\section*{Motivation}
Detecting RNA secondary structures is a process which is integral to understanding the functions of non-coding RNA sequences. Though there are existing bioinformatics tools used for detection, these programs, thwy are not without their limitations.  Specifically,  RNAz 1.0 uses a sliding window approach in its analysis of a sequence [1], meaning there may be structures with loops existing outside the scope of the window. 

The Lempel-Ziv-Welch(LZW) algorithm is a universal lossless compression algorithm used for a variety of applications such as Unix file compression. The algorithm compresses data by using a dictionary to assign common sequences of characters to a fixed-length code (typically 12-bit) [2].

Given the LZW algorithm's dictionary approach to pattern-matching, we hypothesize it could be used as a novel method of detecting new RNA secondary structures which are undetectable by the sliding window method used in RNAz 1.0. 
\section*{Objectives}
Given the time limitations on this project, we intend to focus solely on detection of possible structures without commenting on the structure's validity in nature.  Our goal is to detect RNA secondary structures which may be impossible to detect using RNAz 1.0 using our novel method of detection, RNA\_LZW. 

In addition, we wish to achieve detection in a reasonable time frame, which will be assessed by comparing RNA\_LZW's processing time to that of other RNA secondary structure detection tools. 
\section*{Methods}

code probably

\section*{References}

[1] \url{https://psb.stanford.edu/psb-online/proceedings/psb10/gruber.pdf}

[2] \url{https://www2.cs.duke.edu/courses/spring03/cps296.5/papers/welch_1984_technique_for.pdf}

\end{document}
